\documentclass[10pt, compress]{beamer}

\usetheme{m}

\usepackage{booktabs}
\usepackage[scale=2]{ccicons}
\usepackage{minted}
\usepackage{graphicx}


\usepgfplotslibrary{dateplot}

\usemintedstyle{trac}
\graphicspath{ {images/} }
\DeclareGraphicsExtensions{.pdf,.png,.jpg}
 
\title{Ansible and Cumulus Linux}
\subtitle{}
\date{\today}
\author{Dominik Bay}
\institute{rrbone, Dortmund \& grandcentrix, Cologne}

\begin{document}

\maketitle

\begin{frame}[fragile]
  \frametitle{Ansible and Cumulus Linux}

  The combination of Ansible and Cumulus Linux enables you to fully automate
  your datacenter infrastructure.

  You only need three things to get started
      \begin{itemize}
        \item Ansible \item Cumulus Linux \item A supported network switch
      \end{itemize}

  As you all know Ansible by now, I'll talk about Cumulus Linux and the
  appropiate hardware. Ansible is "only" the tool to keep it all together.

\end{frame}

\section{Cumulus Linux Overview}

\begin{frame}[fragile]
  \frametitle{What is Cumulus Linux?}

Cumulus Linux 
      \begin{itemize}
		\item is a generic Switch Operating System like IOS (Cisco), JunOS (Juniper) and others
		\item runs on industry standard hardware
		\item is based on Debian
      \end{itemize}

Basically like a Linux server with many hardware-accelerated Ethernet ports
provided by the high-performance switch silicon.

It provides bash, iproute2 and other well-known tools.

Licensed on a per instance basis with a very simple licensing model. Switches continue working after the license expires.
Everything is kept on-site, no third party services are needed to use the switch with Cumulus Linux.

\end{frame}

\section{Network Hardware}

\begin{frame}[fragile]
  \frametitle{Where can I use it?}
  
Cumulus Linux supports many 1G, 10G and 40G switches.
Each switch is based on Broadcom Firebolt, Triumph, Apollo and Trident silicon.

DELL and Quanta are probably the most well-known vendors, but also Penguin Computing,
Agema and Edge-corE are in the list.

The complete HCL is available at
\url{cumulusnetworks.com/support/linux-hardware-compatibility-list/}
\end{frame}

\section{How does it work?}

\begin{frame}[fragile]
  \frametitle{How does it work?}

ONIE (Open Network Installation Environment) is Bootloader which is iPXE on steroids.
It was created by Cumulus Linux in 2012 and is part of the Open Compute Project since 2013.

It supports auto-discovery of installation servers by DHCP and HTTP, as well automatic image installation
and provisioning.
Other uses are diagnostics and rescue \& recovery.

More info can be found at \url{opencomputeproject.github.io/onie/docs/index.html}
\end{frame}

\section{Topology}

\begin{frame}[fragile]
  \frametitle{What does our topology look like?}

For the demo we have two switches available. Those are DELL S6000-ON with 32 x 40G-QSFP+ ports.

We're going to

      \begin{itemize}
        \item install the switches via Zero Touch Provisioning in the lab
        \item run a playbook to configure the Prescriptive Topology Manager
        \item setup iBGP between the switches
       \end{itemize}

\begin{center}
  \includegraphics[scale=0.5]{ibgp-test}
\end{center}

\end{frame}

\section{Demo}

\begin{frame}{Super fancy demo}
  \begin{figure}
    \begin{tikzpicture}
      \begin{axis}[
        mlineplot,
        width=0.9\textwidth,
        height=6cm,
      ]

        \addplot {sin(deg(x))};
        \addplot+[samples=50] {sin(deg(4*x))};

      \end{axis}
    \end{tikzpicture}
  \end{figure}
\end{frame}

\plain{Questions?}

\begin{frame}{Summary}

  Get the playbooks I used and the slides at

  \begin{center}\url{github.com/eimann/ansible-cl-dus}\end{center}

  This presentation is licensed under
  \href{http://creativecommons.org/licenses/by-sa/4.0/}{Creative Commons
  Attribution-ShareAlike 4.0 International License}.

  \begin{center}\ccbysa\end{center}
  
  Proudly made with \href{https://github.com/matze/mtheme}{mtheme} and \LaTeX{}! 

\end{frame}

\end{document}
